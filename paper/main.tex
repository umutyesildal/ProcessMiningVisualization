\documentclass[11pt,a4paper]{article}
\usepackage[utf8]{inputenc}
\usepackage[english]{babel}
\usepackage{amsmath}
\usepackage{amsfonts}
\usepackage{amssymb}
\usepackage{graphicx}
\usepackage{booktabs}
\usepackage{hyperref}
\usepackage{cite}
\usepackage[margin=2.5cm]{geometry}
\usepackage{float}
\usepackage{subfigure}
\usepackage{listings}
\usepackage{xcolor}
\usepackage{url}

% Code listing style
\lstset{
    basicstyle=\ttfamily\footnotesize,
    breaklines=true,
    frame=single,
    language=Python,
    showstringspaces=false,
    tabsize=2
}

% Hyperref setup
\hypersetup{
    colorlinks=true,
    linkcolor=blue,
    filecolor=magenta,      
    urlcolor=cyan,
    citecolor=red,
}

\title{Visualizing Event Type Distributions Over Time: Interactive Violin Charts for Process Mining Analysis}

\author{
    Umut Yesildal \\
    \textit{Humboldt-Universität zu Berlin} \\
    \textit{Institut für Informatik} \\
    \textit{Supervisor: Prof. Dr. Jan Mendling} \\
    \textit{Task 10: Event type distribution over time axis} \\
    \texttt{umut.yesildal@student.hu-berlin.de}
}

\date{\today}

\begin{document}

\maketitle

\begin{abstract}
Process mining event logs contain rich temporal information that reveals process behavior patterns. However, traditional visualization methods struggle to effectively display the temporal distributions of event types, particularly when dealing with highly skewed data typical in process mining datasets. This paper presents an interactive dashboard that utilizes violin charts to visualize event type distributions over time, enabling process analysts to understand temporal patterns through multiple statistical sorting parameters. The solution addresses the challenge of visualizing time-since-case-start distributions by implementing smart filtering techniques and providing seven different time transformation methods. The approach is validated across four diverse process mining datasets from healthcare, government, and finance domains, demonstrating cross-domain applicability and scalability from 15,000 to over 1.2 million events. The interactive dashboard supports real-time analysis with configurable sorting options including minimum, maximum, mean, median, and quartile-based arrangements, providing comprehensive insights into process temporal behavior.
\end{abstract}

\section{Introduction}
\label{sec:introduction}

Process mining has emerged as a crucial discipline for understanding and optimizing business processes through the analysis of event logs \cite{aalst2016process}. A fundamental aspect of process analysis involves understanding the temporal distribution of events within process instances, particularly how different event types are distributed over time relative to the case start. This temporal analysis provides valuable insights into process efficiency, bottlenecks, and behavioral patterns.

This work addresses Task 10: \textit{Event type distribution over time axis}, which focuses on calculating when different event types occur on a time axis relative to the start event of sequences, sorting event types based on statistical parameters, and utilizing violin charts as the primary visual element with interactive sorting capabilities \cite{chen2015survey}.

Traditional visualization approaches for temporal process data, such as histograms and scatter plots, often fail to adequately represent the complex distributions typical in process mining datasets. These datasets frequently exhibit extreme skewness, with the majority of events clustering near the case start time, making it difficult to discern meaningful patterns in the temporal distribution.

The challenge is compounded by the presence of case-start events (events occurring at time = 0) that dominate visualizations but provide limited insights into process temporal behavior. Furthermore, different process domains exhibit varying temporal characteristics that require flexible analytical approaches to reveal domain-specific patterns.

This paper addresses these challenges by presenting an interactive dashboard that leverages violin charts for visualizing event type distributions over time. The main contributions of this work include:

\begin{itemize}
    \item A smart filtering approach that removes uninformative case-start events while preserving meaningful temporal data
    \item An interactive visualization framework using violin charts with configurable statistical sorting parameters
    \item Cross-domain validation across healthcare, government, and finance process datasets
    \item A comprehensive transformation system supporting seven different time scaling methods
\end{itemize}
\section{Research Problem and Requirements}
\label{sec:problem}

\subsection{Problem Statement}

Process mining event logs typically contain temporal data that presents several visualization challenges:

\textbf{Extreme Temporal Skewness:} Most process mining datasets exhibit heavy right-tail distributions where 80-90\% of events cluster within the first few time units of case execution, making traditional visualizations ineffective.

\textbf{Case-Start Event Dominance:} Events occurring at time = 0 (case initiation events) create massive spikes in visualizations but provide no temporal insights, obscuring meaningful patterns in subsequent events.

\textbf{Cross-Domain Variability:} Different process domains (healthcare, finance, government) exhibit distinct temporal characteristics requiring adaptable analytical approaches.

\textbf{Statistical Analysis Complexity:} Understanding temporal distributions requires multiple statistical perspectives (mean, median, quartiles) that traditional visualizations cannot effectively combine.

\subsection{Requirements Analysis}

Based on the identified challenges, the following requirements were established:

\textbf{R1 - Smart Data Filtering:} The system must intelligently filter out uninformative case-start events while preserving temporal analysis validity.

\textbf{R2 - Interactive Visualization:} Provide real-time, interactive visualization capabilities using violin charts to show both distribution shape and statistical summaries.

\textbf{R3 - Configurable Sorting:} Enable users to sort event types based on different statistical parameters (minimum, maximum, mean, median, quartiles) for comprehensive analysis, as specified in the task requirements.

\textbf{R4 - Multi-Dataset Support:} Support seamless analysis across multiple process mining datasets from different domains.

\textbf{R5 - Scalability:} Maintain performance and usability across datasets ranging from thousands to millions of events.

\textbf{R6 - Transformation Flexibility:} Provide multiple time transformation methods to handle different types of temporal skewness.

\section{Design and Implementation}
\label{sec:design}

\subsection{System Architecture}

The solution implements a modular architecture comprising four main components:

\textbf{Data Processing Layer:} Handles XES file ingestion, time calculation, and smart filtering using the pm4py library for process mining standard compliance.

\textbf{Transformation Engine:} Implements seven transformation methods including logarithmic scaling, raw time conversion, square root transformation, and min-max scaling to handle different distribution characteristics.

\textbf{Visualization Engine:} Built using Dash and Plotly, provides interactive violin charts with real-time updates and statistical overlay capabilities.

\textbf{User Interface Layer:} Responsive web-based dashboard with sidebar controls for dataset selection, transformation options, sorting parameters, and event count configuration.

\subsection{Smart Filtering Algorithm}

The core innovation lies in the smart filtering approach that addresses the case-start event problem:

\begin{lstlisting}[language=Python, caption=Smart Filtering Implementation]
def load_dataset(dataset_key, num_events=6):
    df = pd.read_csv(data_path)
    
    # Smart filtering: Remove case-start events
    df_filtered = df[df['time_since_case_start'] > 0].copy()
    
    # Select top temporal events for analysis
    top_events = df_filtered['concept:name'].value_counts().head(num_events).index.tolist()
    
    return df_filtered, top_events, dataset_info
\end{lstlisting}

This approach filters out events where \texttt{time\_since\_case\_start = 0}, focusing analysis on events that occur after case initiation and provide meaningful temporal insights.

\subsection{Transformation Framework}

Seven transformation methods were implemented to handle different temporal distribution characteristics:

\begin{itemize}
    \item \textbf{Logarithmic:} $\log(hours + 1)$ for highly skewed data
    \item \textbf{Raw Time:} Hours, days, weeks, months for absolute timing
    \item \textbf{Square Root:} $\sqrt{hours}$ for moderate compression
    \item \textbf{Min-Max Scaling:} $(x - min) / (max - min)$ for cross-process comparison
\end{itemize}

\subsection{Interactive Visualization Design}

The violin chart implementation combines distribution visualization with statistical analysis:

\begin{lstlisting}[language=Python, caption=Violin Chart Generation]
fig = px.violin(
    df_final, 
    x='transformed_time', 
    y='concept:name',
    orientation='h',
    box=True,
    title=plot_title,
    category_orders={'concept:name': event_order}
)
\end{lstlisting}

The horizontal orientation was chosen to accommodate event type labels, while the integrated box plots provide statistical summaries alongside distribution shapes.

\subsection{Statistical Sorting Implementation}

Users can sort event types based on seven statistical parameters:

\begin{itemize}
    \item Frequency (event count)
    \item Mean time
    \item Median time
    \item Minimum time
    \item Maximum time
    \item 25th percentile (Q1)
    \item 75th percentile (Q3)
\end{itemize}

This multi-parameter sorting enables comprehensive temporal analysis from different statistical perspectives.

\section{Evaluation and Discussion}
\label{sec:evaluation}

\subsection{Dataset Evaluation}

The approach was validated across four diverse process mining datasets:

\textbf{Traffic Fines (Government):} 150,370 cases, 561,470 events spanning up to 12 years, demonstrating long-term citizen interaction patterns.

\textbf{BPI Challenge 2012 (Finance):} 13,087 cases, 262,000 events from Dutch financial processes, showing business workflow characteristics.

\textbf{BPI Challenge 2017 (Finance):} 31,509 cases, 1,202,267 events representing complex credit application processes.

\textbf{Sepsis Cases (Healthcare):} 1,050 cases, 15,214 events from hospital patient treatment, exhibiting time-critical medical decision patterns.

\subsection{Smart Filtering Effectiveness}

The smart filtering approach demonstrated significant effectiveness across all datasets:

\begin{table}[H]
\centering
\caption{Smart Filtering Results}
\begin{tabular}{@{}lccc@{}}
\toprule
Dataset & Original Events & After Filtering & Reduction \\
\midrule
Traffic Fines & 561,470 & 400,659 & 28.7\% \\
BPI 2012 & 262,000 & 249,000 & 5.0\% \\
BPI 2017 & 1,202,267 & 1,171,758 & 2.5\% \\
Sepsis Cases & 15,214 & 14,164 & 6.9\% \\
\bottomrule
\end{tabular}
\end{table}

The results show that filtering effectively removes uninformative events while preserving analytical validity, with reduction rates varying based on process characteristics.

\subsection{Cross-Domain Pattern Discovery}

The violin chart visualization revealed distinct temporal patterns across domains:

\textbf{Government Processes:} Exhibited bimodal distributions with early payment clusters and long-tail delayed responses, indicating citizen behavior variability.

\textbf{Financial Processes:} Showed predictable temporal stages corresponding to business process steps, with clear separation between different approval phases.

\textbf{Healthcare Processes:} Demonstrated time-critical clustering with rapid initial responses followed by gradual treatment progression.

\subsection{Performance Analysis}

The system demonstrated excellent scalability and performance:

\begin{itemize}
    \item Real-time dataset switching in under 2 seconds
    \item Consistent performance across 15K to 1.2M events
    \item Interactive transformations completing in under 1 second
    \item Responsive design across desktop, tablet, and mobile devices
\end{itemize}

\subsection{User Experience Validation}

The interactive dashboard successfully addresses the identified requirements:

\textbf{R1 - Smart Filtering:} Achieved 2.5-28.7\% event reduction while preserving temporal insights.

\textbf{R2 - Interactive Visualization:} Violin charts effectively show distribution shapes and statistical summaries simultaneously.

\textbf{R3 - Configurable Sorting:} Seven sorting parameters provide comprehensive analytical perspectives.

\textbf{R4 - Multi-Dataset Support:} Seamless switching between four diverse datasets demonstrates broad applicability.

\textbf{R5 - Scalability:} Consistent performance across four orders of magnitude in dataset size.

\textbf{R6 - Transformation Flexibility:} Seven transformation methods accommodate different temporal characteristics.

\section{Conclusion}
\label{sec:conclusion}

This paper presents a novel approach to visualizing event type distributions over time in process mining through interactive violin charts. The key innovation lies in the smart filtering technique that addresses the fundamental problem of case-start event dominance in temporal visualizations.

The solution demonstrates several important contributions:

\textbf{Methodological Innovation:} The smart filtering approach provides a systematic solution to the case-start event problem, applicable across diverse process mining contexts.

\textbf{Technical Excellence:} The modular architecture and interactive dashboard provide a production-ready tool for process mining analysis.

\textbf{Cross-Domain Validation:} Successful application across government, finance, and healthcare domains demonstrates broad applicability and scalability.

\textbf{Research Impact:} The violin chart approach reveals temporal patterns invisible to traditional visualization methods, enabling new insights into process behavior.

Future work could extend this approach to comparative analysis between multiple datasets simultaneously, temporal pattern clustering, and integration with predictive process mining techniques.

The open-source implementation ensures reproducibility and provides a foundation for further research in temporal process mining visualization.

\bibliographystyle{plain}
\bibliography{references}

\end{document}
