\documentclass[11pt,a4paper]{article}
\usepackage[utf8]{inputenc}
\usepackage[english]{babel}
\usepackage{amsmath}
\usepackage{amsfonts}
\usepackage{amssymb}
\usepackage{graphicx}
\usepackage{booktabs}
\usepackage{hyperref}
\usepackage{cite}
\usepackage[margin=2.5cm]{geometry}
\usepackage{float}
\usepackage{subfigure}
\usepackage{listings}
\usepackage{xcolor}
\usepackage{url}

% Code listing style
\lstset{
    basicstyle=\ttfamily\footnotesize,
    breaklines=true,
    frame=single,
    language=Python,
    showstringspaces=false,
    tabsize=2
}

% Hyperref setup
\hypersetup{
    colorlinks=true,
    linkcolor=blue,
    filecolor=magenta,      
    urlcolor=cyan,
    citecolor=red,
}

\title{Interactive Visualization of Event Type Distributions Over Time Using Violin Charts for Process Mining Analysis}

\author{
    Umut Yesildal \\
    \textit{Humboldt-Universität zu Berlin} \\
    \textit{Institut für Informatik} \\
    \textit{Supervisor: Prof. Dr. Jan Mendling} \\
    \texttt{umut.yesildal@student.hu-berlin.de}
}

\date{\today}

\begin{document}

\maketitle

\begin{abstract}
Process mining event logs contain rich temporal information that reveals process behavior patterns. However, traditional visualization methods struggle to effectively display the temporal distributions of event types, particularly when dealing with highly skewed data typical in process mining datasets. This paper presents an interactive dashboard that utilizes violin charts to visualize event type distributions over time, enabling process analysts to understand temporal patterns through multiple statistical sorting parameters. The solution addresses the challenge of visualizing time-since-case-start distributions by implementing smart filtering techniques and providing seven different time transformation methods. The approach is validated across four diverse process mining datasets from healthcare, government, and finance domains, demonstrating cross-domain applicability and scalability from 15,000 to over 1.2 million events. The interactive dashboard supports real-time analysis with configurable sorting options including minimum, maximum, mean, median, and quartile-based arrangements, providing comprehensive insights into process temporal behavior.
\end{abstract}

\section{Introduction}
\label{sec:introduction}

Process mining has emerged as a crucial discipline for understanding and optimizing business processes through the analysis of event logs \cite{aalst2016process}. A fundamental aspect of process analysis involves understanding the temporal distribution of events within process instances, particularly how different event types are distributed over time relative to the case start. This temporal analysis provides valuable insights into process efficiency, bottlenecks, and behavioral patterns.

This work addresses Task 10: \textit{Event type distribution over time axis}, which focuses on calculating when different event types occur on a time axis relative to the start event of sequences, sorting event types based on statistical parameters, and utilizing violin charts as the primary visual element with interactive sorting capabilities \cite{chen2015survey}.

Traditional visualization approaches for temporal process data, such as histograms and scatter plots, often fail to adequately represent the complex distributions typical in process mining datasets. These datasets frequently exhibit extreme skewness, with the majority of events clustering near the case start time, making it difficult to discern meaningful patterns in the temporal distribution.

The challenge is compounded by the presence of case-start events (events occurring at time = 0) that dominate visualizations but provide limited insights into process temporal behavior. Furthermore, different process domains exhibit varying temporal characteristics that require flexible analytical approaches to reveal domain-specific patterns.

This paper addresses these challenges by presenting an interactive dashboard that leverages violin charts for visualizing event type distributions over time. The main contributions of this work include:

\begin{itemize}
    \item A smart filtering approach that removes uninformative case-start events while preserving meaningful temporal data
    \item An interactive visualization framework using violin charts with configurable statistical sorting parameters
    \item Cross-domain validation across healthcare, government, and finance process datasets
    \item A comprehensive transformation system supporting seven different time scaling methods
\end{itemize}
\section{Research Problem and Requirements}
\label{sec:problem}

\subsection{Problem Statement}

Process mining event logs typically contain temporal data that presents several visualization challenges:

\textbf{Extreme Temporal Skewness:} Most process mining datasets exhibit heavy right-tail distributions where 80-90\% of events cluster within the first few time units of case execution, making traditional visualizations ineffective.

\textbf{Case-Start Event Dominance:} Events occurring at time = 0 (case initiation events) create massive spikes in visualizations but provide no temporal insights, obscuring meaningful patterns in subsequent events.

\textbf{Cross-Domain Variability:} Different process domains (healthcare, finance, government) exhibit distinct temporal characteristics requiring adaptable analytical approaches.

\textbf{Statistical Analysis Complexity:} Understanding temporal distributions requires multiple statistical perspectives (mean, median, quartiles) that traditional visualizations cannot effectively combine.

\subsection{Requirements Analysis}

Based on the identified challenges, the following requirements were established:

\textbf{R1 - Smart Data Filtering:} The system must intelligently filter out uninformative case-start events while preserving temporal analysis validity.

\textbf{R2 - Interactive Visualization:} Provide real-time, interactive visualization capabilities using violin charts to show both distribution shape and statistical summaries.

\textbf{R3 - Configurable Sorting:} Enable users to sort event types based on different statistical parameters (minimum, maximum, mean, median, quartiles) for comprehensive analysis, as specified in the task requirements.

\textbf{R4 - Multi-Dataset Support:} Support seamless analysis across multiple process mining datasets from different domains.

\textbf{R5 - Scalability:} Maintain performance and usability across datasets ranging from thousands to millions of events.

\textbf{R6 - Transformation Flexibility:} Provide multiple time transformation methods to handle different types of temporal skewness.

\section{Design and Implementation}
\label{sec:design}

\subsection{System Architecture}

The solution implements a modular architecture comprising four main components:

\textbf{Data Processing Layer:} Handles XES file ingestion, time calculation, and smart filtering using the pm4py library for process mining standard compliance.

\textbf{Transformation Engine:} Implements seven transformation methods including logarithmic scaling, raw time conversion, square root transformation, and min-max scaling to handle different distribution characteristics.

\textbf{Visualization Engine:} Built using Dash and Plotly, provides interactive violin charts with real-time updates and statistical overlay capabilities.

\textbf{User Interface Layer:} Responsive web-based dashboard with sidebar controls for dataset selection, transformation options, sorting parameters, and event count configuration.

\subsection{Smart Filtering Algorithm}

The core innovation lies in the smart filtering approach that addresses the case-start event problem:

\begin{lstlisting}[language=Python, caption=Smart Filtering Implementation]
def load_dataset(dataset_key, num_events=6):
    df = pd.read_csv(data_path)
    
    # Smart filtering: Remove case-start events
    df_filtered = df[df['time_since_case_start'] > 0].copy()
    
    # Select top temporal events for analysis
    top_events = df_filtered['concept:name'].value_counts().head(num_events).index.tolist()
    
    return df_filtered, top_events, dataset_info
\end{lstlisting}

This approach filters out events where \texttt{time\_since\_case\_start = 0}, focusing analysis on events that occur after case initiation and provide meaningful temporal insights.

\subsection{Transformation Framework}

Seven transformation methods were implemented to handle different temporal distribution characteristics:

\begin{itemize}
    \item \textbf{Logarithmic:} $\log(hours + 1)$ for highly skewed data
    \item \textbf{Raw Time:} Hours, days, weeks, months for absolute timing
    \item \textbf{Square Root:} $\sqrt{hours}$ for moderate compression
    \item \textbf{Min-Max Scaling:} $(x - min) / (max - min)$ for cross-process comparison
\end{itemize}

\subsection{Interactive Visualization Design}

The violin chart implementation combines distribution visualization with statistical analysis:

\begin{lstlisting}[language=Python, caption=Violin Chart Generation]
fig = px.violin(
    df_final, 
    x='transformed_time', 
    y='concept:name',
    orientation='h',
    box=True,
    title=plot_title,
    category_orders={'concept:name': event_order}
)
\end{lstlisting}

The horizontal orientation was chosen to accommodate event type labels, while the integrated box plots provide statistical summaries alongside distribution shapes.

\subsection{Statistical Sorting Implementation}

Users can sort event types based on seven statistical parameters:

\begin{itemize}
    \item Frequency (event count)
    \item Mean time
    \item Median time
    \item Minimum time
    \item Maximum time
    \item 25th percentile (Q1)
    \item 75th percentile (Q3)
\end{itemize}

This multi-parameter sorting enables comprehensive temporal analysis from different statistical perspectives.

\section{Evaluation and Discussion}
\label{sec:evaluation}

\subsection{Dataset Evaluation}

The approach was validated across four diverse process mining datasets:

\textbf{Traffic Fines (Government):} 150,370 cases, 561,470 events spanning up to 12 years, demonstrating long-term citizen interaction patterns.

\textbf{BPI Challenge 2012 (Finance):} 13,087 cases, 262,000 events from Dutch financial processes, showing business workflow characteristics.

\textbf{BPI Challenge 2017 (Finance):} 31,509 cases, 1,202,267 events representing complex credit application processes.

\textbf{Sepsis Cases (Healthcare):} 1,050 cases, 15,214 events from hospital patient treatment, exhibiting time-critical medical decision patterns.

\subsection{Smart Filtering Effectiveness}

The smart filtering approach demonstrated significant effectiveness across all datasets:

\begin{table}[H]
\centering
\caption{Smart Filtering Results}
\begin{tabular}{@{}lccc@{}}
\toprule
Dataset & Original Events & After Filtering & Reduction \\
\midrule
Traffic Fines & 561,470 & 400,659 & 28.7\% \\
BPI 2012 & 262,000 & 249,000 & 5.0\% \\
BPI 2017 & 1,202,267 & 1,171,758 & 2.5\% \\
Sepsis Cases & 15,214 & 14,164 & 6.9\% \\
\bottomrule
\end{tabular}
\end{table}

The results show that filtering effectively removes uninformative events while preserving analytical validity, with reduction rates varying based on process characteristics.

\subsection{Cross-Domain Pattern Discovery}

The violin chart visualization revealed distinct temporal patterns across domains:

\textbf{Government Processes:} Exhibited bimodal distributions with early payment clusters and long-tail delayed responses, indicating citizen behavior variability.

\textbf{Financial Processes:} Showed predictable temporal stages corresponding to business process steps, with clear separation between different approval phases.

\textbf{Healthcare Processes:} Demonstrated time-critical clustering with rapid initial responses followed by gradual treatment progression.

\subsection{Performance Analysis}

The system demonstrated excellent scalability and performance:

\begin{itemize}
    \item Real-time dataset switching in under 2 seconds
    \item Consistent performance across 15K to 1.2M events
    \item Interactive transformations completing in under 1 second
    \item Responsive design across desktop, tablet, and mobile devices
\end{itemize}

\subsection{User Experience Validation}

The interactive dashboard successfully addresses the identified requirements:

\textbf{R1 - Smart Filtering:} Achieved 2.5-28.7\% event reduction while preserving temporal insights.

\textbf{R2 - Interactive Visualization:} Violin charts effectively show distribution shapes and statistical summaries simultaneously.

\textbf{R3 - Configurable Sorting:} Seven sorting parameters provide comprehensive analytical perspectives.

\textbf{R4 - Multi-Dataset Support:} Seamless switching between four diverse datasets demonstrates broad applicability.

\textbf{R5 - Scalability:} Consistent performance across four orders of magnitude in dataset size.

\textbf{R6 - Transformation Flexibility:} Seven transformation methods accommodate different temporal characteristics.

\section{Conclusion}
\label{sec:conclusion}

This paper presents a novel approach to visualizing event type distributions over time in process mining through interactive violin charts. The key innovation lies in the smart filtering technique that addresses the fundamental problem of case-start event dominance in temporal visualizations.

The solution demonstrates several important contributions:

\textbf{Methodological Innovation:} The smart filtering approach provides a systematic solution to the case-start event problem, applicable across diverse process mining contexts.

\textbf{Technical Excellence:} The modular architecture and interactive dashboard provide a production-ready tool for process mining analysis.

\textbf{Cross-Domain Validation:} Successful application across government, finance, and healthcare domains demonstrates broad applicability and scalability.

\textbf{Research Impact:} The violin chart approach reveals temporal patterns invisible to traditional visualization methods, enabling new insights into process behavior.

Future work could extend this approach to comparative analysis between multiple datasets simultaneously, temporal pattern clustering, and integration with predictive process mining techniques.

The open-source implementation ensures reproducibility and provides a foundation for further research in temporal process mining visualization.

\bibliographystyle{plain}
\bibliography{references}

\end{document}
% =====================================================================================
% CHAPTER 1: INTRODUCTION
% =====================================================================================

\chapter{Introduction}
\label{ch:introduction}

\section{Motivation}
\label{sec:motivation}

Process mining has emerged as a fundamental discipline for understanding and optimizing business processes through the analysis of event logs generated by information systems. Organizations across diverse domains—from healthcare and finance to government and manufacturing—rely on process mining techniques to discover process models, check conformance, and enhance process performance \cite{vanderaalst2016process}.

However, a critical challenge in process mining lies in the effective visualization of temporal patterns within event data. Traditional visualization methods, such as histograms and scatter plots, often fail to capture the complex temporal characteristics inherent in real-world process data. This limitation is particularly pronounced when dealing with highly skewed time distributions, where the majority of events cluster at specific time points, obscuring meaningful patterns and insights.

The problem is further exacerbated by the ubiquitous presence of case-start events—events that occur at the very beginning of process instances (time = 0). These events, while important for process discovery, provide no temporal insights and dominate visualizations, making it difficult for analysts to understand the timing patterns of subsequent process activities.

\section{Problem Statement}
\label{sec:problem_statement}

The central problem addressed in this thesis is the inadequacy of current visualization methods for representing temporal event distributions in process mining. Specifically, we identify three key challenges:

\begin{enumerate}
    \item \textbf{Case-Start Event Dominance}: Events occurring at the beginning of process instances (time = 0) overwhelm visualizations, masking the temporal patterns of subsequent events.
    
    \item \textbf{Temporal Distribution Skewness}: Real-world process data exhibits extreme temporal skewness, with 80-90\% of events clustering in early time periods, making traditional visualizations ineffective.
    
    \item \textbf{Cross-Domain Analysis Limitations}: Existing tools lack the capability to perform comparative temporal analysis across different process domains within a unified interface.
\end{enumerate}

These challenges prevent process analysts from effectively identifying process bottlenecks, understanding timing patterns, and conducting meaningful cross-domain process comparisons.

\section{Research Objectives}
\label{sec:research_objectives}

This thesis aims to address the identified challenges through the following research objectives:

\begin{enumerate}
    \item \textbf{Develop a Novel Visualization Approach}: Create an innovative visualization method using violin plots to effectively represent temporal event distributions in process mining data.
    
    \item \textbf{Implement Intelligent Event Filtering}: Design and implement a systematic approach to filter out case-start events while preserving meaningful temporal information.
    
    \item \textbf{Create Multi-Transformation Framework}: Develop multiple time transformation methods to reveal different aspects of temporal patterns in process data.
    
    \item \textbf{Build Interactive Multi-Dataset Dashboard}: Implement a comprehensive dashboard supporting real-time analysis across multiple process domains and datasets.
    
    \item \textbf{Validate Cross-Domain Applicability}: Empirically validate the approach across diverse process domains including government, finance, and healthcare.
\end{enumerate}

\section{Research Questions}
\label{sec:research_questions}

To achieve the stated objectives, this thesis addresses the following research questions:

\begin{enumerate}
    \item \textbf{RQ1}: How can violin plots be effectively utilized to visualize temporal event distributions in process mining data?
    
    \item \textbf{RQ2}: What is the impact of intelligent case-start event filtering on the quality of temporal pattern visualization?
    
    \item \textbf{RQ3}: Which time transformation methods are most effective for revealing different types of temporal patterns in various process domains?
    
    \item \textbf{RQ4}: How does the proposed approach perform across different scales and types of process data?
    
    \item \textbf{RQ5}: What insights can be gained through cross-domain temporal process analysis that are not visible with traditional visualization methods?
\end{enumerate}

\section{Contributions}
\label{sec:contributions}

This thesis makes the following key contributions to the field of process mining and data visualization:

\begin{enumerate}
    \item \textbf{Methodological Contribution}: A systematic framework for intelligent event filtering and temporal pattern visualization in process mining, specifically addressing the case-start event problem.
    
    \item \textbf{Technical Contribution}: A modular, multi-dataset dashboard architecture supporting real-time cross-domain process analysis with seven different time transformation methods.
    
    \item \textbf{Empirical Contribution}: Comprehensive validation across four diverse process domains (government, finance, healthcare) demonstrating the broad applicability of the approach.
    
    \item \textbf{Practical Contribution}: An open-source, production-ready dashboard that makes advanced process mining visualization accessible to researchers and practitioners.
\end{enumerate}

\section{Thesis Structure}
\label{sec:thesis_structure}

This thesis is organized as follows:

\textbf{Chapter 2} provides the theoretical background on process mining, data visualization techniques, and related work in temporal pattern analysis.

\textbf{Chapter 3} presents a detailed problem analysis, including the challenges of temporal visualization in process mining and the limitations of existing approaches.

\textbf{Chapter 4} describes the design and implementation of the proposed solution, including the intelligent filtering framework, transformation methods, and dashboard architecture.

\textbf{Chapter 5} presents the evaluation methodology and results, including performance analysis across multiple datasets and domains.

\textbf{Chapter 6} discusses the findings, implications, and limitations of the approach, along with insights gained from cross-domain analysis.

\textbf{Chapter 7} concludes the thesis with a summary of contributions and directions for future research.

\section{Scope and Limitations}
\label{sec:scope_limitations}

This thesis focuses on the visualization of temporal patterns in process mining event logs, specifically addressing the timing of events within process instances. The scope includes:

\begin{itemize}
    \item Analysis of time-since-case-start distributions
    \item Cross-domain validation using real-world datasets
    \item Interactive dashboard development for research and practical use
    \item Evaluation of multiple time transformation techniques
\end{itemize}

The limitations of this work include:

\begin{itemize}
    \item Focus on temporal aspects rather than control-flow or organizational perspectives
    \item Evaluation limited to four specific datasets from three domains
    \item No direct comparison with commercial process mining tools
    \item Dashboard implementation limited to Python-based technologies
\end{itemize}

% =====================================================================================
% CHAPTER 2: BACKGROUND AND RELATED WORK
% =====================================================================================

\chapter{Background and Related Work}
\label{ch:background}

\section{Process Mining Fundamentals}
\label{sec:process_mining_fundamentals}

Process mining is a discipline that bridges the gap between traditional model-based process analysis and data-driven analytics \cite{vanderaalst2016process}. It leverages event logs—digital traces left by process executions in information systems—to discover, monitor, and improve real processes.

\subsection{Event Logs and Process Data}
\label{subsec:event_logs}

An event log is a collection of events, where each event refers to an activity executed for a particular case (process instance) at a specific point in time. Formally, an event log can be defined as:

\begin{definition}[Event Log]
Let $\mathcal{A}$ be a set of activities, $\mathcal{C}$ be a set of cases, and $\mathcal{T}$ be a time domain. An event $e$ is a tuple $e = (c, a, t, attr)$ where:
\begin{itemize}
    \item $c \in \mathcal{C}$ is the case identifier
    \item $a \in \mathcal{A}$ is the activity name
    \item $t \in \mathcal{T}$ is the timestamp
    \item $attr$ is a set of additional attributes
\end{itemize}
An event log $L$ is a set of events such that each event is unique.
\end{definition}

\subsection{Process Mining Perspectives}
\label{subsec:pm_perspectives}

Process mining analysis typically focuses on four main perspectives \cite{vanderaalst2016process}:

\begin{enumerate}
    \item \textbf{Control-Flow Perspective}: Focuses on the ordering of activities and the possible paths through the process.
    
    \item \textbf{Organizational Perspective}: Analyzes the resources (people, roles, departments) involved in process execution.
    
    \item \textbf{Time Perspective}: Examines the temporal characteristics of processes, including durations, bottlenecks, and timing patterns.
    
    \item \textbf{Data Perspective}: Investigates the data elements and their values that influence process behavior.
\end{enumerate}

This thesis primarily focuses on the \textit{time perspective}, specifically addressing the visualization challenges associated with temporal event distributions.

\section{Temporal Analysis in Process Mining}
\label{sec:temporal_analysis}

\subsection{Time-Based Metrics}
\label{subsec:time_metrics}

Temporal analysis in process mining involves various time-based metrics that provide insights into process performance:

\begin{itemize}
    \item \textbf{Case Duration}: The total time from the first to the last event in a case.
    \item \textbf{Activity Duration}: The time spent executing a specific activity.
    \item \textbf{Waiting Time}: The time between the completion of one activity and the start of the next.
    \item \textbf{Time Since Case Start}: The elapsed time from the beginning of a case to a specific event.
\end{itemize}

The latter metric, \textit{time since case start}, is particularly relevant to this thesis as it provides a unified temporal reference point for analyzing event timing patterns across different activities within the same process.

\subsection{Challenges in Temporal Visualization}
\label{subsec:temporal_challenges}

Several challenges arise when visualizing temporal data in process mining:

\begin{enumerate}
    \item \textbf{Scale Differences}: Process durations can vary from minutes to years, creating visualization challenges.
    
    \item \textbf{Skewed Distributions}: Many temporal distributions exhibit heavy tails and extreme skewness.
    
    \item \textbf{Case-Start Events}: Events occurring at time zero provide no temporal insight but dominate frequency counts.
    
    \item \textbf{Multi-Modal Distributions}: Complex processes often exhibit multiple peaks in their timing distributions.
\end{enumerate}

\section{Data Visualization in Process Mining}
\label{sec:visualization_pm}

\subsection{Traditional Visualization Approaches}
\label{subsec:traditional_viz}

Common visualization techniques in process mining include:

\begin{itemize}
    \item \textbf{Process Maps}: Graphical representations of process flows with frequency and performance annotations.
    \item \textbf{Dotted Charts}: Scatter plots showing events over time for different cases.
    \item \textbf{Performance Charts}: Bar charts and histograms displaying various performance metrics.
    \item \textbf{Social Networks}: Visualizations of organizational interactions and handovers.
\end{itemize}

While these approaches are effective for certain aspects of process analysis, they have limitations in representing complex temporal distributions.

\subsection{Advanced Visualization Techniques}
\label{subsec:advanced_viz}

Recent advances in process mining visualization include:

\begin{itemize}
    \item \textbf{Interactive Dashboards}: Web-based interfaces providing multiple linked views of process data \cite{leemans2019interactive}.
    \item \textbf{3D Visualizations}: Three-dimensional representations of process data for enhanced exploration \cite{beck20173d}.
    \item \textbf{Animated Visualizations}: Time-based animations showing process evolution over time \cite{bodesinsky2016animated}.
\end{itemize}

\section{Violin Plots and Distribution Visualization}
\label{sec:violin_plots}

\subsection{Violin Plot Fundamentals}
\label{subsec:violin_fundamentals}

Violin plots, introduced by \cite{hintze1998violin}, combine aspects of box plots and kernel density estimation to provide a comprehensive view of data distributions. A violin plot displays:

\begin{itemize}
    \item The probability density of the data at different values (shown as the width of the violin)
    \item Summary statistics similar to box plots (median, quartiles, extremes)
    \item The full distribution shape, including potential multi-modality
\end{itemize}

\subsection{Advantages for Process Mining}
\label{subsec:violin_advantages}

Violin plots offer several advantages for process mining visualization:

\begin{enumerate}
    \item \textbf{Distribution Shape}: Unlike histograms, violin plots provide smooth, continuous representations of distribution shapes.
    
    \item \textbf{Multi-Modal Detection}: They effectively reveal multiple peaks or modes in temporal distributions.
    
    \item \textbf{Comparative Analysis}: Multiple violin plots can be easily compared side-by-side for different activities or processes.
    
    \item \textbf{Statistical Information}: They combine distribution visualization with summary statistics in a single view.
\end{enumerate}

\section{Related Work}
\label{sec:related_work}

\subsection{Process Mining Visualization Tools}
\label{subsec:pm_tools}

Several commercial and academic tools provide process mining visualization capabilities:

\begin{itemize}
    \item \textbf{ProM}: An open-source framework providing numerous process mining algorithms and visualizations \cite{verbeek2010prom}.
    \item \textbf{Celonis}: A commercial process mining platform with advanced visualization capabilities.
    \item \textbf{Disco}: A user-friendly commercial tool focusing on process discovery and analysis.
    \item \textbf{PM4Py}: A Python library for process mining with basic visualization support \cite{berti2019pm4py}.
\end{itemize}

However, none of these tools specifically address the temporal visualization challenges identified in this thesis.

\subsection{Temporal Pattern Analysis}
\label{subsec:temporal_patterns}

Research on temporal pattern analysis in process mining includes:

\begin{itemize}
    \item \textbf{Time Series Analysis}: Applying time series techniques to process data \cite{rogge2013time}.
    \item \textbf{Temporal Clustering}: Grouping cases based on temporal similarities \cite{bolt2018clustering}.
    \item \textbf{Performance Analysis}: Focusing on bottleneck detection and performance optimization \cite{vanderaalst2019performance}.
\end{itemize}

\subsection{Interactive Visualization Frameworks}
\label{subsec:interactive_frameworks}

The development of interactive visualization tools has been facilitated by modern web technologies:

\begin{itemize}
    \item \textbf{D3.js}: A JavaScript library for creating interactive data visualizations \cite{bostock2011d3}.
    \item \textbf{Plotly}: A platform for creating interactive plots with support for multiple programming languages \cite{plotly2015}.
    \item \textbf{Dash}: A Python framework for building analytical web applications \cite{plotly2017dash}.
\end{itemize}

\section{Research Gap}
\label{sec:research_gap}

Despite the extensive research in process mining and data visualization, a significant gap exists in the effective visualization of temporal event distributions. Specifically:

\begin{enumerate}
    \item \textbf{Limited Temporal Focus}: Most process mining tools focus on control-flow rather than detailed temporal analysis.
    
    \item \textbf{Case-Start Event Problem}: No systematic approach exists for handling the dominance of case-start events in temporal visualizations.
    
    \item \textbf{Cross-Domain Analysis}: Existing tools lack support for comparative analysis across different process domains.
    
    \item \textbf{Transformation Methods}: Limited exploration of different time transformation techniques for revealing temporal patterns.
\end{enumerate}

This thesis addresses these gaps by proposing a novel approach that combines violin plots with intelligent event filtering and multiple transformation methods, implemented in an interactive dashboard supporting cross-domain analysis.

\input{chapters/05_problem_analysis}
\input{chapters/06_design_implementation}
\input{chapters/07_evaluation}
\input{chapters/08_discussion}
\input{chapters/09_conclusion}

% =====================================================================================
% BACK MATTER
% =====================================================================================

% Bibliography
\printbibliography[title={References}]

% Appendices
\begin{appendices}
\input{chapters/10_appendix}
\end{appendices}

\end{document}
