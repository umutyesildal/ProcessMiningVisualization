% =====================================================================================
% CHAPTER 1: INTRODUCTION
% =====================================================================================

\chapter{Introduction}
\label{ch:introduction}

\section{Motivation}
\label{sec:motivation}

Process mining has emerged as a fundamental discipline for understanding and optimizing business processes through the analysis of event logs generated by information systems. Organizations across diverse domains—from healthcare and finance to government and manufacturing—rely on process mining techniques to discover process models, check conformance, and enhance process performance \cite{vanderaalst2016process}.

However, a critical challenge in process mining lies in the effective visualization of temporal patterns within event data. Traditional visualization methods, such as histograms and scatter plots, often fail to capture the complex temporal characteristics inherent in real-world process data. This limitation is particularly pronounced when dealing with highly skewed time distributions, where the majority of events cluster at specific time points, obscuring meaningful patterns and insights.

The problem is further exacerbated by the ubiquitous presence of case-start events—events that occur at the very beginning of process instances (time = 0). These events, while important for process discovery, provide no temporal insights and dominate visualizations, making it difficult for analysts to understand the timing patterns of subsequent process activities.

\section{Problem Statement}
\label{sec:problem_statement}

The central problem addressed in this thesis is the inadequacy of current visualization methods for representing temporal event distributions in process mining. Specifically, we identify three key challenges:

\begin{enumerate}
    \item \textbf{Case-Start Event Dominance}: Events occurring at the beginning of process instances (time = 0) overwhelm visualizations, masking the temporal patterns of subsequent events.
    
    \item \textbf{Temporal Distribution Skewness}: Real-world process data exhibits extreme temporal skewness, with 80-90\% of events clustering in early time periods, making traditional visualizations ineffective.
    
    \item \textbf{Cross-Domain Analysis Limitations}: Existing tools lack the capability to perform comparative temporal analysis across different process domains within a unified interface.
\end{enumerate}

These challenges prevent process analysts from effectively identifying process bottlenecks, understanding timing patterns, and conducting meaningful cross-domain process comparisons.

\section{Research Objectives}
\label{sec:research_objectives}

This thesis aims to address the identified challenges through the following research objectives:

\begin{enumerate}
    \item \textbf{Develop a Novel Visualization Approach}: Create an innovative visualization method using violin plots to effectively represent temporal event distributions in process mining data.
    
    \item \textbf{Implement Intelligent Event Filtering}: Design and implement a systematic approach to filter out case-start events while preserving meaningful temporal information.
    
    \item \textbf{Create Multi-Transformation Framework}: Develop multiple time transformation methods to reveal different aspects of temporal patterns in process data.
    
    \item \textbf{Build Interactive Multi-Dataset Dashboard}: Implement a comprehensive dashboard supporting real-time analysis across multiple process domains and datasets.
    
    \item \textbf{Validate Cross-Domain Applicability}: Empirically validate the approach across diverse process domains including government, finance, and healthcare.
\end{enumerate}

\section{Research Questions}
\label{sec:research_questions}

To achieve the stated objectives, this thesis addresses the following research questions:

\begin{enumerate}
    \item \textbf{RQ1}: How can violin plots be effectively utilized to visualize temporal event distributions in process mining data?
    
    \item \textbf{RQ2}: What is the impact of intelligent case-start event filtering on the quality of temporal pattern visualization?
    
    \item \textbf{RQ3}: Which time transformation methods are most effective for revealing different types of temporal patterns in various process domains?
    
    \item \textbf{RQ4}: How does the proposed approach perform across different scales and types of process data?
    
    \item \textbf{RQ5}: What insights can be gained through cross-domain temporal process analysis that are not visible with traditional visualization methods?
\end{enumerate}

\section{Contributions}
\label{sec:contributions}

This thesis makes the following key contributions to the field of process mining and data visualization:

\begin{enumerate}
    \item \textbf{Methodological Contribution}: A systematic framework for intelligent event filtering and temporal pattern visualization in process mining, specifically addressing the case-start event problem.
    
    \item \textbf{Technical Contribution}: A modular, multi-dataset dashboard architecture supporting real-time cross-domain process analysis with seven different time transformation methods.
    
    \item \textbf{Empirical Contribution}: Comprehensive validation across four diverse process domains (government, finance, healthcare) demonstrating the broad applicability of the approach.
    
    \item \textbf{Practical Contribution}: An open-source, production-ready dashboard that makes advanced process mining visualization accessible to researchers and practitioners.
\end{enumerate}

\section{Thesis Structure}
\label{sec:thesis_structure}

This thesis is organized as follows:

\textbf{Chapter 2} provides the theoretical background on process mining, data visualization techniques, and related work in temporal pattern analysis.

\textbf{Chapter 3} presents a detailed problem analysis, including the challenges of temporal visualization in process mining and the limitations of existing approaches.

\textbf{Chapter 4} describes the design and implementation of the proposed solution, including the intelligent filtering framework, transformation methods, and dashboard architecture.

\textbf{Chapter 5} presents the evaluation methodology and results, including performance analysis across multiple datasets and domains.

\textbf{Chapter 6} discusses the findings, implications, and limitations of the approach, along with insights gained from cross-domain analysis.

\textbf{Chapter 7} concludes the thesis with a summary of contributions and directions for future research.

\section{Scope and Limitations}
\label{sec:scope_limitations}

This thesis focuses on the visualization of temporal patterns in process mining event logs, specifically addressing the timing of events within process instances. The scope includes:

\begin{itemize}
    \item Analysis of time-since-case-start distributions
    \item Cross-domain validation using real-world datasets
    \item Interactive dashboard development for research and practical use
    \item Evaluation of multiple time transformation techniques
\end{itemize}

The limitations of this work include:

\begin{itemize}
    \item Focus on temporal aspects rather than control-flow or organizational perspectives
    \item Evaluation limited to four specific datasets from three domains
    \item No direct comparison with commercial process mining tools
    \item Dashboard implementation limited to Python-based technologies
\end{itemize}
