% =====================================================================================
% SECTION 5: CONCLUSION
% =====================================================================================

\section{Conclusion}
\label{sec:conclusion}

This paper presents a comprehensive approach to visualizing event type distributions over time in process mining through interactive violin charts. Our implementation successfully addresses key challenges in temporal event analysis by combining statistical summaries with distribution shapes in a single, interpretable visualization.

\subsection{Key Contributions}
\label{subsec:contributions}

Our work makes several important contributions to process mining visualization:

\textbf{Integrated Visualization Design:} We developed violin charts that simultaneously display distribution shapes and statistical summaries, eliminating the need to switch between multiple visualization types. This integration enables analysts to identify patterns such as bimodal distributions while maintaining access to quantitative measures.

\textbf{Flexible Time Transformation Framework:} Our approach provides multiple transformation options (logarithmic, square root, raw, and min-max scaling) that adapt to different dataset characteristics. This flexibility proves essential as no single transformation works optimally across all process domains and timing scales.

\textbf{Cross-Domain Pattern Discovery:} Testing across government, finance, and healthcare processes revealed distinct temporal signatures for each domain. Government processes exhibit regulatory deadline patterns, financial processes show structured workflow stages, and healthcare processes demonstrate time-critical clustering.

\textbf{Event Filtering:} Our case-start event filtering preserves analytical validity while removing noise that obscures meaningful patterns. The filtering effectiveness varies by process structure, with simpler processes showing higher filtering rates.

\textbf{Interactive Statistical Sorting:} Multiple sorting options (frequency, mean, median) provide complementary analytical perspectives from the same data, enabling comprehensive exploration of timing patterns.

\subsection{Practical Impact}
\label{subsec:impact}

The evaluation demonstrates significant practical benefits over traditional visualization approaches:

\textbf{Pattern Recognition:} Complex patterns like bimodal distributions in government processes and time-critical clustering in healthcare are clearly visible, information that remains hidden in box plots or histograms.

\textbf{Process Understanding:} The visualization reveals domain-specific timing behaviors, such as citizen payment patterns in traffic fines (30-day and 6-month peaks) and clinical protocol adherence in sepsis treatment.

\textbf{Analytical Efficiency:} Single-view access to both statistical and distributional information reduces cognitive load and analysis time compared to examining multiple separate charts.

\textbf{Scalability:} The approach handles datasets ranging from 15,000 to 1.2 million events while maintaining interactive performance and visual clarity.

\subsection{Limitations and Future Work}
\label{subsec:future}

Several limitations suggest directions for future research:


\textbf{Sparse Data Visualization:} Event types with very few occurrences can create misleading violin shapes. Advanced smoothing techniques or alternative visualization methods for sparse data merit investigation.

\textbf{Scalability Boundaries:} While tested up to 1.2 million events, performance with datasets exceeding 10 million events remains unexplored and may require additional optimization strategies.

\textbf{Comparative Analysis Tools:} Enhanced support for cross-dataset comparison could provide deeper insights into process mining patterns across different domains and organizations.

\subsection{Final Remarks}
\label{subsec:remarks}

Our approach demonstrates that violin charts provide a powerful tool for temporal event analysis in process mining. The combination of distribution visualization, statistical integration, and flexible transformation makes complex timing patterns accessible to analysts across diverse domains. The cross-domain evaluation validates the approach's generalizability while revealing domain-specific insights that inform both research and practice.

The implementation provides a solid foundation for future process mining visualization research, particularly in temporal analysis and cross-domain pattern discovery. As process mining continues to expand across industries, visualization tools that adapt to diverse timing characteristics while maintaining analytical rigor become increasingly important for extracting actionable insights from complex event data.
