% =====================================================================================
% CHAPTER 2: BACKGROUND AND RELATED WORK
% =====================================================================================

\chapter{Background and Related Work}
\label{ch:background}

\section{Process Mining Fundamentals}
\label{sec:process_mining_fundamentals}

Process mining is a discipline that bridges the gap between traditional model-based process analysis and data-driven analytics \cite{vanderaalst2016process}. It leverages event logs—digital traces left by process executions in information systems—to discover, monitor, and improve real processes.

\subsection{Event Logs and Process Data}
\label{subsec:event_logs}

An event log is a collection of events, where each event refers to an activity executed for a particular case (process instance) at a specific point in time. Formally, an event log can be defined as:

\begin{definition}[Event Log]
Let $\mathcal{A}$ be a set of activities, $\mathcal{C}$ be a set of cases, and $\mathcal{T}$ be a time domain. An event $e$ is a tuple $e = (c, a, t, attr)$ where:
\begin{itemize}
    \item $c \in \mathcal{C}$ is the case identifier
    \item $a \in \mathcal{A}$ is the activity name
    \item $t \in \mathcal{T}$ is the timestamp
    \item $attr$ is a set of additional attributes
\end{itemize}
An event log $L$ is a set of events such that each event is unique.
\end{definition}

\subsection{Process Mining Perspectives}
\label{subsec:pm_perspectives}

Process mining analysis typically focuses on four main perspectives \cite{vanderaalst2016process}:

\begin{enumerate}
    \item \textbf{Control-Flow Perspective}: Focuses on the ordering of activities and the possible paths through the process.
    
    \item \textbf{Organizational Perspective}: Analyzes the resources (people, roles, departments) involved in process execution.
    
    \item \textbf{Time Perspective}: Examines the temporal characteristics of processes, including durations, bottlenecks, and timing patterns.
    
    \item \textbf{Data Perspective}: Investigates the data elements and their values that influence process behavior.
\end{enumerate}

This thesis primarily focuses on the \textit{time perspective}, specifically addressing the visualization challenges associated with temporal event distributions.

\section{Temporal Analysis in Process Mining}
\label{sec:temporal_analysis}

\subsection{Time-Based Metrics}
\label{subsec:time_metrics}

Temporal analysis in process mining involves various time-based metrics that provide insights into process performance:

\begin{itemize}
    \item \textbf{Case Duration}: The total time from the first to the last event in a case.
    \item \textbf{Activity Duration}: The time spent executing a specific activity.
    \item \textbf{Waiting Time}: The time between the completion of one activity and the start of the next.
    \item \textbf{Time Since Case Start}: The elapsed time from the beginning of a case to a specific event.
\end{itemize}

The latter metric, \textit{time since case start}, is particularly relevant to this thesis as it provides a unified temporal reference point for analyzing event timing patterns across different activities within the same process.

\subsection{Challenges in Temporal Visualization}
\label{subsec:temporal_challenges}

Several challenges arise when visualizing temporal data in process mining:

\begin{enumerate}
    \item \textbf{Scale Differences}: Process durations can vary from minutes to years, creating visualization challenges.
    
    \item \textbf{Skewed Distributions}: Many temporal distributions exhibit heavy tails and extreme skewness.
    
    \item \textbf{Case-Start Events}: Events occurring at time zero provide no temporal insight but dominate frequency counts.
    
    \item \textbf{Multi-Modal Distributions}: Complex processes often exhibit multiple peaks in their timing distributions.
\end{enumerate}

\section{Data Visualization in Process Mining}
\label{sec:visualization_pm}

\subsection{Traditional Visualization Approaches}
\label{subsec:traditional_viz}

Common visualization techniques in process mining include:

\begin{itemize}
    \item \textbf{Process Maps}: Graphical representations of process flows with frequency and performance annotations.
    \item \textbf{Dotted Charts}: Scatter plots showing events over time for different cases.
    \item \textbf{Performance Charts}: Bar charts and histograms displaying various performance metrics.
    \item \textbf{Social Networks}: Visualizations of organizational interactions and handovers.
\end{itemize}

While these approaches are effective for certain aspects of process analysis, they have limitations in representing complex temporal distributions.

\subsection{Advanced Visualization Techniques}
\label{subsec:advanced_viz}

Recent advances in process mining visualization include:

\begin{itemize}
    \item \textbf{Interactive Dashboards}: Web-based interfaces providing multiple linked views of process data \cite{leemans2019interactive}.
    \item \textbf{3D Visualizations}: Three-dimensional representations of process data for enhanced exploration \cite{beck20173d}.
    \item \textbf{Animated Visualizations}: Time-based animations showing process evolution over time \cite{bodesinsky2016animated}.
\end{itemize}

\section{Violin Plots and Distribution Visualization}
\label{sec:violin_plots}

\subsection{Violin Plot Fundamentals}
\label{subsec:violin_fundamentals}

Violin plots, introduced by \cite{hintze1998violin}, combine aspects of box plots and kernel density estimation to provide a comprehensive view of data distributions. A violin plot displays:

\begin{itemize}
    \item The probability density of the data at different values (shown as the width of the violin)
    \item Summary statistics similar to box plots (median, quartiles, extremes)
    \item The full distribution shape, including potential multi-modality
\end{itemize}

\subsection{Advantages for Process Mining}
\label{subsec:violin_advantages}

Violin plots offer several advantages for process mining visualization:

\begin{enumerate}
    \item \textbf{Distribution Shape}: Unlike histograms, violin plots provide smooth, continuous representations of distribution shapes.
    
    \item \textbf{Multi-Modal Detection}: They effectively reveal multiple peaks or modes in temporal distributions.
    
    \item \textbf{Comparative Analysis}: Multiple violin plots can be easily compared side-by-side for different activities or processes.
    
    \item \textbf{Statistical Information}: They combine distribution visualization with summary statistics in a single view.
\end{enumerate}

\section{Related Work}
\label{sec:related_work}

\subsection{Process Mining Visualization Tools}
\label{subsec:pm_tools}

Several commercial and academic tools provide process mining visualization capabilities:

\begin{itemize}
    \item \textbf{ProM}: An open-source framework providing numerous process mining algorithms and visualizations \cite{verbeek2010prom}.
    \item \textbf{Celonis}: A commercial process mining platform with advanced visualization capabilities.
    \item \textbf{Disco}: A user-friendly commercial tool focusing on process discovery and analysis.
    \item \textbf{PM4Py}: A Python library for process mining with basic visualization support \cite{berti2019pm4py}.
\end{itemize}

However, none of these tools specifically address the temporal visualization challenges identified in this thesis.

\subsection{Temporal Pattern Analysis}
\label{subsec:temporal_patterns}

Research on temporal pattern analysis in process mining includes:

\begin{itemize}
    \item \textbf{Time Series Analysis}: Applying time series techniques to process data \cite{rogge2013time}.
    \item \textbf{Temporal Clustering}: Grouping cases based on temporal similarities \cite{bolt2018clustering}.
    \item \textbf{Performance Analysis}: Focusing on bottleneck detection and performance optimization \cite{vanderaalst2019performance}.
\end{itemize}

\subsection{Interactive Visualization Frameworks}
\label{subsec:interactive_frameworks}

The development of interactive visualization tools has been facilitated by modern web technologies:

\begin{itemize}
    \item \textbf{D3.js}: A JavaScript library for creating interactive data visualizations \cite{bostock2011d3}.
    \item \textbf{Plotly}: A platform for creating interactive plots with support for multiple programming languages \cite{plotly2015}.
    \item \textbf{Dash}: A Python framework for building analytical web applications \cite{plotly2017dash}.
\end{itemize}

\section{Research Gap}
\label{sec:research_gap}

Despite the extensive research in process mining and data visualization, a significant gap exists in the effective visualization of temporal event distributions. Specifically:

\begin{enumerate}
    \item \textbf{Limited Temporal Focus}: Most process mining tools focus on control-flow rather than detailed temporal analysis.
    
    \item \textbf{Case-Start Event Problem}: No systematic approach exists for handling the dominance of case-start events in temporal visualizations.
    
    \item \textbf{Cross-Domain Analysis}: Existing tools lack support for comparative analysis across different process domains.
    
    \item \textbf{Transformation Methods}: Limited exploration of different time transformation techniques for revealing temporal patterns.
\end{enumerate}

This thesis addresses these gaps by proposing a novel approach that combines violin plots with intelligent event filtering and multiple transformation methods, implemented in an interactive dashboard supporting cross-domain analysis.
